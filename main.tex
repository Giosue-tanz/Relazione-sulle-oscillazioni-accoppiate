\documentclass{article}
\input{preambolo.sty}

\title{Relazione oscillazioni accopiata}
\author{Aiello Giosuè, Fenili Domenico, Sermi Francesco}
\date{\today}

\begin{document}
\maketitle
\newpage
\tableofcontents
\section{Scopo}
Studiare il moto di due pendoli accoppiati e del fenomeno dei \emph{battimenti}
\section{Premesse teoriche}
qualcosa che si scrive subito dopo

\section{Descrizione delle misure}

Per effettuare questa esperienza era necessario misurare il periodo di oscillazione delle oscillazioni accoppiate
in varie situazioni. \\
Prima di procedere però con le misurazioni di questi, abbiamo misurato la lunghezza dell'asta che oscillava e la sua larghezza, che risulta essere pari a
\begin{equation*}
	L =
\end{equation*}
e abbiamo misurato il raggio e lo spessore del disco a cui l'asta era attaccata, pari a
\begin{equation}
	R = 
\end{equation}
\textbf{Osservazioni}: sebbene l'asta presentasse dei \emph{buchi}, abbiamo considerato come trascurabile lo spostamento del centro di massa dovuto a questi. Inoltre l'asta e il disco si sono considerati omogenei, quindi la loro densità era costante in ogni punto del materiale. \\
Dopo aver effettuato questo, abbiamo iniziato a misurare l'oscillazione del singolo pendolo prima senza lo smorzatore e poi senza lo smorzatore: ai fini di queste misure, abbiamo usufruito di un programma di acquisizione che girava sul PC che avevamo a disposizione in laboratorio che andava a stimare la posizione del pendolo in $u.a$ (unità arbitrarie) in base alla variazione di corrente dovuta al passaggio del pendolo nell'acqua lì presente. \\
Dopo aver effettuato queste misure, siamo passati ad analizzare il moto del pendolo se accoppiato ad un altro oscillatore armonico tramite oscillazioni secondo i modi normali: per fare ciò abbiamo utilizzato una molla con cui abbiamo collegato i due pendoli e, successivamente, abbiamo 
\end{document}